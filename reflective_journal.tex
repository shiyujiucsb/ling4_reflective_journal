\documentclass{article}

\usepackage{fullpage}

\begin{document}
\title{LING 4: Reflective Journal}
\author{Shiyu Ji}
\date{}
\maketitle

\begin{itemize}

\newcommand{\question}[1]{\fbox{\parbox{0.93\textwidth}{#1}}}

\item
\question{
What class activities and tasks were the most helpful to you, 
and in what ways did they contribute to the development of your skills?
}

{\bf Answer.} I think giving little lessons and simulating small lectures were the most beneficial for me. They gave me a great opportunity to practice speaking publicly, which is extremely important to me, since there are a lot of talks, lectures and academic meetings waiting for me in the future.

By practicing talking in front of the whiteboard, I came to understand that if I want to tell the story clearly, then I must firstly fully understand the story by myself. Also it is important to maintain eye contact during the talk. To attract the audience, it is also very important to give stress on speech rather than keep a monotone.

\item
\question{
What did you do outside of “class” to supplement the practice in class during the quarter?
}

{\bf Answer.} I listened to Rachel's English, which gives great help on English pronunciation. By following Rachel's talks, I came to know a lot of details in English that I have not learned before.

I also talked to Shannon Li weekly. We basically talked about campus life, many cultural differences, etc. Shannon is very helpful on improving my English fluency. I also learned a lot how to communicate with young people, e.g., undergraduate students.

Usually I listen to many online lectures and academic talks, which are usually given by great experts, e.g., MIT's OpenCourseWare, Winter School at Bar-Ilan University, etc. Even though many speakers there are not native English speakers, they well reflect the situation in academia (especially engineering fields) nowadays.

\item 
\question{
Please discuss any plans you have for continuing to develop your skills in English after the class ends.
}

{\bf Answer.} Keep listening to different English resources, e.g., online classes, Rachel's English, NPR Podcasts, TED, etc. Also there are extensive opportunities to speak with professors and students on campus. There are also a lot of events, e.g., academic talks, conferences, lectures, in which I can practice my English listening and speaking.

Also it is a great idea to make friends with some English native speakers. 

\item 
\question{
Do you have any suggestions for improving the course in the future?
}

{\bf Answer.} Basically this course is very helpful for a beginning PhD student. 

My suggestion is that maybe it can be better to have some buzzwords for different cases. For example, when lecturing, professors frequently say ``you see'', ``make sure we are on the same page'', ``let me put it in this way'', etc.

\end{itemize}

\end{document}
